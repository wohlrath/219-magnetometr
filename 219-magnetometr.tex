\documentclass[a4paper]{article}

\usepackage[czech]{babel} %https://github.com/michal-h21/biblatex-iso690
\usepackage[
   backend=biber      % if we want unicode 
  ,style=iso-numeric % or iso-numeric for numeric citation method          
  ,babel=other        % to support multiple languages in bibliography
  ,sortlocale=cs_CZ   % locale of main language, it is for sorting
  ,bibencoding=UTF8   % this is necessary only if bibliography file is in different encoding than main document
]{biblatex}

\usepackage[utf8]{inputenc}
\usepackage{fancyhdr}
\usepackage{amsmath}
\usepackage{amssymb}
\usepackage[left=2cm,right=2cm,top=2.5cm,bottom=2.5cm]{geometry}
\usepackage{graphicx}
\usepackage{pdfpages}
\usepackage{url}

\usepackage{siunitx}
\sisetup{locale = DE}  %, separate-uncertainty = true    kdybych chtel +/-

\usepackage{float}
\newfloat{graph}{htbp}{grp}
\floatname{graph}{Graf}
\newfloat{tabulka}{htbp}{tbl}
\floatname{tabulka}{Tabulka}

\renewcommand{\thefootnote}{\roman{footnote}}

\pagestyle{fancy}
\lhead{Praktikum II - (19) Měření s torzním magnetometrem}
\rhead{Vladislav Wohlrath}
\author{Vladislav Wohlrath}

\bibliography{source}

\begin{document}

\begin{titlepage}
\includepdf[pages={1}]{./graficos/219-tit.pdf}
\end{titlepage}

\section*{Pracovní úkoly}
\begin{enumerate}
\item Změřte závislost výchylky magnetometru na proudu protékajícím cívkou. Měření proveďte pro obě cívky a různé počty závitů (\num{5} a \num{10}).
\item Výsledky měření znázorněte graficky.
\item Diskutujte výsledky měření z hlediska platnosti Biot-Savartova zákona.
\item Změřte direkční moment vlákna metodou torzních kmitů.
\item Určete magnetický moment magnetu užívaného při měření (v Coulombových i Ampérových jednotkách). 
\end{enumerate}

%Teoretická část
\section*{Teoretická část}

Malý permanentní tyčový magnet o neznámém Coulombově magnetickém momentu $p$ zavěsíme vodorovně na tenké vlákno a umístíme do středu kruhové cívky kolmo k jeho ose.
Pokud bude cívka mít poloměr $r$, počet závitů $N$ a poteče jí proud $I$, vytvoří v místě magnetu podle Biotova-Savartova\cite{elektrika} zákona magnetické pole o~intenzitě
\begin{equation}
H=\frac{NI}{2r} \,.
\end{equation}
Vektor intenzity pole bude kolmý na magnetický moment magnetu a na magnet bude působit moment síly
\begin{equation}
M=pH \,,
\end{equation}
a vychýlí se z původní polohy o~úhel\footnote{Platí pro malé úhly}
\begin{equation} \label{eq:vzorecalfa}
\alpha = \frac{M}{D} = \frac{pH}{D} = \frac{pNI}{2rD} \,,
\end{equation}
kde $D$ je direkční moment vlákna.
Z Biotova-Savartova zákona tedy vyplývá závislost
\begin{equation} \label{eq:zavislostalfa}
\alpha \propto \frac{NI}{r} \,,
\end{equation}
kterou experimentálně ověříme.

Direkční moment $D$ určíme metodou torzních kmitů.
Na vlákno zavěsíme vodorovně mosaznou tyč.
Jestliže je moment setrvačnosti tyče vzhledem k ose otáčení $J$ a zanedbáme momenty ostatních částí magnetometru, bude kyvadlo kmitat s periodou 
\begin{equation} \label{eq:periodadirekcnimoment}
T=2\pi \sqrt{\frac{J}{D}} \,.
\end{equation}

Ze známého direkčního momentu a naměřené závislosti \eqref{eq:zavislostalfa} můžeme pomocí \eqref{eq:vzorecalfa} vypočítat magnetický moment $p$.

Kromě Coulombova magnetického momentu $p$ definujeme též Ampérův magnetický moment
\begin{equation} \label{eq:amper}
m=\frac{p}{\mu_0}
\end{equation}

%Výsledky měření
\section*{Výsledky měření}

Pokud není uvedeno jinak, uvedené odchylky jsou standardní a odchylku nepřímo měřených veličin určujeme metodou přenosu chyby.
Používáme zápis $x=\SI{10(1)}{\cm}$, kde číslo v závorce vyjadřuje odchylku v řádu poslední uvedené číslice, tedy $x=\SI[separate-uncertainty=true]{10(1)}{\cm}$.


Pokus probíhal při normálním tlaku a pokojové teplotě ($t\approx\SI{22}{\degreeCelsius}$).

Měření jsme provedli se dvěma kruhovými cívkami, které budeme důsledně nazývat větší ($d=2r=\SI{40.5(5)}{\cm}$) a menší ($d=\SI{20.0(2)}{\cm}$).
Obě cívky měly 10 závitů a umožňovaly zapojení, ve kterém tekl proud jen 5 závity.
Měřili jsme tyčový magnet označený jako MAGNET2.

Na magnet jsme připevnili malé zrcadlo a zamířili jsme na něj laserový paprsek.
Do vzdálenosti $L=\SI{1.14(1)}{\m}$ od magnetu jsme umístili stínítko tak, aby na něj v rovnovážné poloze s nulovým proudem cívkou paprsek dopadal kolmo.

Pro obě cívky jsme měnili proud v rozmezí 0--\SI{4}{\ampere} a měřili výchylku místa dopadu laseru na stínítko. Proud jsme měřili vnějším ampérmetrem.
Pokud se vychýlil o $\Delta l$, úhel otočení magnetu určíme jako
\begin{equation}
\alpha = \frac{1}{2} \arctan \frac{\Delta l}{L} \,.
\end{equation}

Přímo hodnoty $\Delta l$ neuvádíme, naměřené úhly jsou uvedeny v tabulce \ref{tab:tabmain} a zaneseny do grafu \ref{graf:grafmain}


\begin{tabulka}[htbp]
\centering
\begin{tabular}{c|c|c|c|c}

 & \multicolumn{2}{c|}{menší cívka} & \multicolumn{2}{c}{větší cívka} \\
  & \multicolumn{2}{c|}{$r = \SI{10(1)}{\cm} $} & \multicolumn{2}{c}{$r= \SI{20.3(3)}{\cm} $} \\
  
  & $N=5$ & $N=10$ & $N=5$ & $N=10$ \\ \hline
  
$I$ (\si{\ampere}) & $\alpha$(\si{\degree}) & $\alpha$(\si{\degree}) & $\alpha$(\si{\degree}) & $\alpha$(\si{\degree}) \\
\hline

\num{0.5} & \num{0.40} & \num{0.80} & \num{0.23} & \num{0.45} \\
\num{1.0} & \num{0.80} & \num{1.63} & \num{0.43} & \num{0.85} \\
\num{1.5} & --- & \num{2.43} & --- & --- \\
\num{2.0} & \num{1.63} & \num{3.23} & \num{0.83} & \num{1.68} \\
\num{2.5} & --- & \num{4.04} & --- & --- \\
\num{3.0} & \num{2.43} & \num{4.83} & \num{1.26} & \num{2.48} \\
\num{3.5} & --- & \num{5.61} & --- & --- \\
\num{4.0} & \num{3.23} & \num{6.38} & \num{1.68} & \num{3.30} \\

\end{tabular}
\caption{Naměřená závislost úhlu otočení magnetu na volbě cívky a proudu jí protékajícím}
\label{tab:tabmain}
\end{tabulka}

\begin{graph}[htbp] 
\centering
% GNUPLOT: LaTeX picture with Postscript
\begingroup
  \makeatletter
  \providecommand\color[2][]{%
    \GenericError{(gnuplot) \space\space\space\@spaces}{%
      Package color not loaded in conjunction with
      terminal option `colourtext'%
    }{See the gnuplot documentation for explanation.%
    }{Either use 'blacktext' in gnuplot or load the package
      color.sty in LaTeX.}%
    \renewcommand\color[2][]{}%
  }%
  \providecommand\includegraphics[2][]{%
    \GenericError{(gnuplot) \space\space\space\@spaces}{%
      Package graphicx or graphics not loaded%
    }{See the gnuplot documentation for explanation.%
    }{The gnuplot epslatex terminal needs graphicx.sty or graphics.sty.}%
    \renewcommand\includegraphics[2][]{}%
  }%
  \providecommand\rotatebox[2]{#2}%
  \@ifundefined{ifGPcolor}{%
    \newif\ifGPcolor
    \GPcolortrue
  }{}%
  \@ifundefined{ifGPblacktext}{%
    \newif\ifGPblacktext
    \GPblacktextfalse
  }{}%
  % define a \g@addto@macro without @ in the name:
  \let\gplgaddtomacro\g@addto@macro
  % define empty templates for all commands taking text:
  \gdef\gplbacktext{}%
  \gdef\gplfronttext{}%
  \makeatother
  \ifGPblacktext
    % no textcolor at all
    \def\colorrgb#1{}%
    \def\colorgray#1{}%
  \else
    % gray or color?
    \ifGPcolor
      \def\colorrgb#1{\color[rgb]{#1}}%
      \def\colorgray#1{\color[gray]{#1}}%
      \expandafter\def\csname LTw\endcsname{\color{white}}%
      \expandafter\def\csname LTb\endcsname{\color{black}}%
      \expandafter\def\csname LTa\endcsname{\color{black}}%
      \expandafter\def\csname LT0\endcsname{\color[rgb]{1,0,0}}%
      \expandafter\def\csname LT1\endcsname{\color[rgb]{0,1,0}}%
      \expandafter\def\csname LT2\endcsname{\color[rgb]{0,0,1}}%
      \expandafter\def\csname LT3\endcsname{\color[rgb]{1,0,1}}%
      \expandafter\def\csname LT4\endcsname{\color[rgb]{0,1,1}}%
      \expandafter\def\csname LT5\endcsname{\color[rgb]{1,1,0}}%
      \expandafter\def\csname LT6\endcsname{\color[rgb]{0,0,0}}%
      \expandafter\def\csname LT7\endcsname{\color[rgb]{1,0.3,0}}%
      \expandafter\def\csname LT8\endcsname{\color[rgb]{0.5,0.5,0.5}}%
    \else
      % gray
      \def\colorrgb#1{\color{black}}%
      \def\colorgray#1{\color[gray]{#1}}%
      \expandafter\def\csname LTw\endcsname{\color{white}}%
      \expandafter\def\csname LTb\endcsname{\color{black}}%
      \expandafter\def\csname LTa\endcsname{\color{black}}%
      \expandafter\def\csname LT0\endcsname{\color{black}}%
      \expandafter\def\csname LT1\endcsname{\color{black}}%
      \expandafter\def\csname LT2\endcsname{\color{black}}%
      \expandafter\def\csname LT3\endcsname{\color{black}}%
      \expandafter\def\csname LT4\endcsname{\color{black}}%
      \expandafter\def\csname LT5\endcsname{\color{black}}%
      \expandafter\def\csname LT6\endcsname{\color{black}}%
      \expandafter\def\csname LT7\endcsname{\color{black}}%
      \expandafter\def\csname LT8\endcsname{\color{black}}%
    \fi
  \fi
  \setlength{\unitlength}{0.0500bp}%
  \begin{picture}(10204.00,6802.00)%
    \gplgaddtomacro\gplbacktext{%
      \csname LTb\endcsname%
      \put(1078,704){\makebox(0,0)[r]{\strut{} 0}}%
      \csname LTb\endcsname%
      \put(1078,1676){\makebox(0,0)[r]{\strut{} 0.02}}%
      \csname LTb\endcsname%
      \put(1078,2648){\makebox(0,0)[r]{\strut{} 0.04}}%
      \csname LTb\endcsname%
      \put(1078,3621){\makebox(0,0)[r]{\strut{} 0.06}}%
      \csname LTb\endcsname%
      \put(1078,4593){\makebox(0,0)[r]{\strut{} 0.08}}%
      \csname LTb\endcsname%
      \put(1078,5565){\makebox(0,0)[r]{\strut{} 0.1}}%
      \csname LTb\endcsname%
      \put(1078,6537){\makebox(0,0)[r]{\strut{} 0.12}}%
      \csname LTb\endcsname%
      \put(1210,484){\makebox(0,0){\strut{} 0}}%
      \csname LTb\endcsname%
      \put(2165,484){\makebox(0,0){\strut{} 0.5}}%
      \csname LTb\endcsname%
      \put(3120,484){\makebox(0,0){\strut{} 1}}%
      \csname LTb\endcsname%
      \put(4076,484){\makebox(0,0){\strut{} 1.5}}%
      \csname LTb\endcsname%
      \put(5031,484){\makebox(0,0){\strut{} 2}}%
      \csname LTb\endcsname%
      \put(5986,484){\makebox(0,0){\strut{} 2.5}}%
      \csname LTb\endcsname%
      \put(6941,484){\makebox(0,0){\strut{} 3}}%
      \csname LTb\endcsname%
      \put(7897,484){\makebox(0,0){\strut{} 3.5}}%
      \csname LTb\endcsname%
      \put(8852,484){\makebox(0,0){\strut{} 4}}%
      \csname LTb\endcsname%
      \put(9807,484){\makebox(0,0){\strut{} 4.5}}%
      \put(176,3620){\rotatebox{-270}{\makebox(0,0){\strut{}$\alpha$ (\si{\radian})}}}%
      \put(5508,154){\makebox(0,0){\strut{}$I$ (\si{\ampere})}}%
    }%
    \gplgaddtomacro\gplfronttext{%
      \csname LTb\endcsname%
      \put(4642,6364){\makebox(0,0)[r]{\strut{}Malá cívka 5 závitů}}%
      \csname LTb\endcsname%
      \put(4642,6144){\makebox(0,0)[r]{\strut{}Malá cívka 10 závitů}}%
      \csname LTb\endcsname%
      \put(4642,5924){\makebox(0,0)[r]{\strut{}Velká cívka 5 závitů}}%
      \csname LTb\endcsname%
      \put(4642,5704){\makebox(0,0)[r]{\strut{}Velká cívka 10 závitů}}%
    }%
    \gplbacktext
    \put(0,0){\includegraphics{graf}}%
    \gplfronttext
  \end{picture}%
\endgroup

\caption{Naměřená závislost úhlu otočení magnetu na volbě cívky a proudu jí protékající, teoretická závislost \eqref{eq:vzorecalfa} s nafitovaným magnetickým momentem $p$}
\label{graf:grafmain}
\end{graph}


Pro změření direkčního momentu vlákna jsme na něj zavěsili mosaznou tyč o délce $l_t=\SI{24.0(1)}{\cm}$, hmotnosti $m_t=\SI{56.6}{\g}$ a průměru $d_t=\SI{6}{\mm}$.
Moment setrvačnosti takové tyče vzhledem k ose procházející průměrem v~jejím středu je podle \cite{momentset}
\begin{equation} \label{eq:momentsetrvacnosti}
J=m(d_t^2+\frac{1}{12} l_t^2)= \SI{2.74e-4}{\kg\m\squared} \,.
\end{equation}
Změřili jsme dobu 20 kmitů \SI{80.5}{\s}, tedy perioda $T=\SI{4.03}{\s}$.
Direkční moment vlákna jsme určili podle \eqref{eq:periodadirekcnimoment} $D=\num{6.6(2)}\cdot \SI{e-4}{\newton\metre\per\radian}$. Odchylku jsme odhadli na \SI{3}{\percent}.

Naměřenou závislost \eqref{eq:vzorecalfa} jsme nafitovali Coulombovým magnetickým momentem $p = \num{3.8(2)} \cdot \SI{e-7}{\weber \meter}$.
Vzhledem k nepřesnostem při měření jsme odchylku odhadli na \SI{5}{\percent}.
Ampérův magnetický moment vypočítáme podle \eqref{eq:amper} jako $m = \SI{0.30(2)}{\ampere\metre\squared}$.
Teoretickou závislost \eqref{eq:vzorecalfa} jsme zanesli do grafu \ref{graf:grafmain} pro porovnání s naměřenými veličinami.

%Diskuze výsledků
\section*{Diskuze}
Z Biotova-Savartova zákona plyne, že bychom pro jeden konkrétní magnet měli být schopni najít konstantu $p$ tak, že naměřené hodnoty pro obě cívky a všechny proudy budou odpovídat vzorci \eqref{eq:vzorecalfa}.
Takovou konstantu se nám skutečně podařilo najít, naměřené hodnoty vykazují dobrou shodu s teoretickou závislostí (viz graf \ref{graf:grafmain}).
Můžeme soudit, že naše výsledky jsou ve shodě s Biotovým-Savartovým zákonem.

Domnělým zdrojem velkých nepřesností byly otřesy v místnosti, avšak i po důkladném dupání v bezprostřední blízkosti aparatury nebyly naměřeny žádné odchylky.
Další chyby mohly být způsobeny nedokonalým tvarem cívek, nelinearitou vlákna či kolísáním proudu, tyto chyby však považujeme za malé a měření za neobyčejně přesné.

Naopak měření direkčního momentu vlákna považujeme za nepříliš přesné, k měření délky mosazné tyče byl k dispozici pouze svinovací metr a ostatní parametry tyče byly napsané na přiloženém papírku bez údaje o jejich přesnosti.
Kromě toho jsme zanedbávali ostatní části aparatury zavěšené na vlákně.
Ve vzorci \eqref{eq:momentsetrvacnosti} jsme mohli zanedbat první člen.



%Závěr
\section*{Závěr}
Změřili jsme závislost výchylky magnetometru na proudu procházejícím cívkou pro dvě různé cívky a různé počty závitů.
Naměřené hodnoty jsou uvedeny v tabulce \ref{tab:tabmain} a zaneseny do grafu \ref{graf:grafmain}.
Do grafu je též vynesena teoretická závislost vyplývající z Biotova-Savartova zákona.
Hodnoty se dobře shodují, takže Biotův-Savartův zákon zůstává v platnosti.

Dále jsme metodou torzních kmitů změřili direkční moment vlákna v magnetometru $D=\num{6.6(2)}\cdot \SI{e-4}{\newton\metre\per\radian}$.

Z naměřené závislosti jsme určili Coulombův magnetický moment $p = \num{3.8(2)} \cdot \SI{e-7}{\weber \meter}$ nebo též Ampérův magnetický moment $m = \SI{0.30(2)}{\ampere\metre\squared}$.


\printbibliography[title={Seznam použité literatury}]

\end{document}