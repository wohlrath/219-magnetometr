\section*{Teoretická část}

Malý permanentní tyčový magnet o neznámém Coulombově magnetickém momentu $p$ zavěsíme vodorovně na tenké vlákno a umístíme do středu kruhové cívky kolmo k jeho ose.
Pokud bude cívka mít poloměr $r$, počet závitů $N$ a poteče jí proud $I$, vytvoří v místě magnetu podle Biotova-Savartova zákona magnetické pole o intenzitě
\begin{equation}
H=\frac{NI}{2r} \,.
\end{equation}
Vektor intenzity pole bude kolmý na magnetický moment magnetu a na magnet bude působit moment síly
\begin{equation}
M=pH \,,
\end{equation}
a vychýlí se z původní polohy o úhel\footnote{Pro malé úhly, kdy je torzní síla pružná a platí $\sin(\alpha)\approx\alpha$}
\begin{equation} \label{eq:vzorecalfa}
\alpha = \frac{M}{D} = \frac{pH}{D} = \frac{pNI}{2rD} \,,
\end{equation}
kde $D$ je direkční moment vlákna.
Z Biotova-Savartova zákona tedy vyplývá závislost
\begin{equation} \label{eq:zavislostalfa}
\alpha \propto \frac{NI}{r} \,,
\end{equation}
kterou experimentálně ověříme.

Direkční moment $D$ určíme metodou torzních kmitů.
Na vlákno zavěsíme vodorovně mosaznou tyč.
Jestliže je moment setrvačnosti tyče vzhledem k ose otáčení $J$ a zanedbáme momenty ostatních částí magnetometru, bude kyvadlo kmitat s periodou 
\begin{equation}
T=2\pi \sqrt{\frac{J}{D}} \,.
\end{equation}

Ze známého direkčního momentu a naměřené závislosti \eqref{eq:zavislostalfa} můžeme pomocí \eqref{eq:vzorecalfa} vypočítat magnetický moment $p$.

Kromě Coulombova magnetického momentu $p$ definujeme též Ampérův magnetický moment
\begin{equation}
m=\frac{p}{\mu_0}
\end{equation}