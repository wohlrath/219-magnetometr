\section*{Závěr}
Změřili jsme závislost výchylky magnetometru na proudu procházejícím cívkou pro dvě různé cívky a různé počty závitů.
Naměřené hodnoty jsou uvedeny v tabulce \ref{tab:tabmain} a zaneseny do grafu \ref{graf:grafmain}.
Do grafu je též vynesena teoretická závislost vyplývající z Biotova-Savartova zákona.
Hodnoty se dobře shodují, takže Biotův-Savartův zákon zůstává v platnosti.

Dále jsme metodou torzních kmitů změřili direkční moment vlákna v magnetometru $D=\num{6.6(2)}\cdot \SI{e-4}{\newton\metre\per\radian}$.

Z naměřené závislosti jsme určili Coulombův magnetický moment $p = \num{3.8(2)} \cdot \SI{e-7}{\weber \meter}$ nebo též Ampérův magnetický moment $m = \SI{0.30(2)}{\ampere\metre\squared}$.